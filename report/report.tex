\documentclass{jsarticle}
\usepackage[dvipdfmx]{graphicx}
\usepackage{indentfirst}
\usepackage{multirow}
\usepackage{here}
\usepackage{listings, jlisting}

\renewcommand{\lstlistingname}{リスト}
\lstset{language=c,
  numbers=left,
  breaklines=true,
  basicstyle=\scriptsize,
  tabsize=2
}

\title{学習と進化的計算 課題レポート}
\author{C0114265 志太 悠真}

\begin{document}

\maketitle

\tableofcontents

\part{ホップフィールドネットワーク}
\section{プログラム}
\lstinputlisting[caption=Hopfield.c]{../Hopfield/Hopfield.c}

\section{解説}
\input{Hopfield}

\section{実行結果}
\lstinputlisting[caption=Hopfield.cの実行結果]{../Hopfield/output.log}

\section{考察}


\part{誤差逆伝播法}
\section{プログラム}
\lstinputlisting[caption=BP.c]{../BP/BP.c}

\section{解説}
\input{BP}

\section{実行結果}
\lstinputlisting[caption=BP.cの実行結果]{../BP/output.log}

\section{考察}


\part{自己組織化マップ}
\section{プログラム}
\lstinputlisting[caption=SOM.c]{../SOM/SOM.c}

\section{解説}
\input{SOM}

\section{実行結果}
\lstinputlisting[caption=SOM.cの実行結果]{../SOM/output.log}

\section{考察}


\part{遺伝的アルゴリズム}
\section{プログラム}
\lstinputlisting[caption=GA.c]{../GA/GA.c}

\section{解説}
\input{GA}

\section{実行結果}
\lstinputlisting[caption=GA.cの実行結果]{../GA/output.log}

\section{考察}


\end{document}
